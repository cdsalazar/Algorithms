\documentclass[11pt]{article} %Sets the default text size to 11pt and class to article.
\usepackage{amsmath}
\newcommand{\BigO}[1]{\ensuremath{\operatorname{O}\bigl(#1\bigr)}}
%------------------------Dimensions--------------------------------------------
\topmargin=-.5in %length of margin at the top of the page (1 inch added by default)
\oddsidemargin=-0.2in %length of margin on sides for odd pages
\evensidemargin=0in %length of margin on sides for even pages
\textwidth=6.5in %How wide you want your text to be
\marginparwidth=0.5in
\headheight=0pt %1in margins at top and bottom (1 inch is added to this value by default)
\headsep=0pt %Increase to increase white space in between headers and the top of the page
\textheight=10.0in %How tall the text body is allowed to be on each page
\pagestyle{empty}
\begin{document}
\centerline{{ \LARGE \bf Problem Set 4}} 
\centerline{CSCI 3104 Spring 2014} 
\centerline{Cristobal Salazar}
\centerline{07/22}
\centerline{Partner: Alex Tsankov}

\noindent{\Large \bf Problem 1}
\\
\indent{\large a) }
\\   

\indent{\large b){ In the first three elements of the list, we have $a:1, b:1, c:2$. This means that adding the $a:1$ and $b:1$ we will get $ab:2$ which is equal to $c:2$. Because these are equal, we can switch them around and still get the same amount of compression. There are $2^2=4$ combinations of Optimal Huffman Codes depending on how we tie-break. Then we can do the same procedure, but mirror the tree. }
\\

\indent{\large c) }
\\

\noindent{\Large \bf Problem 2}

\indent{\large a) }
\\

\indent{\large b) }
\\

\indent{\large c) }


\noindent{\Large \bf Problem 3}
\\
\indent{\large }
\\


\noindent{\Large \bf Sources}
\\
\indent $\bullet$ http://www.bowdoin.edu/~ltoma/teaching/cs231/fall07/Lectures/reccurences.pdf
%this is for 4c
\\
\indent $\bullet$ http://www.cs.unm.edu/~saia/561-f13/lec/lec3.pdf
%this is for 4b


\end{document}
