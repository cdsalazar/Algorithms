\documentclass[11pt]{article} %Sets the default text size to 11pt and class to article.
\usepackage{amsmath}
\newcommand{\BigO}[1]{\ensuremath{\operatorname{O}\bigl(#1\bigr)}}
%------------------------Dimensions--------------------------------------------
\topmargin=-.5in %length of margin at the top of the page (1 inch added by default)
\oddsidemargin=-0.2in %length of margin on sides for odd pages
\evensidemargin=0in %length of margin on sides for even pages
\textwidth=6.5in %How wide you want your text to be
\marginparwidth=0.5in
\headheight=0pt %1in margins at top and bottom (1 inch is added to this value by default)
\headsep=0pt %Increase to increase white space in between headers and the top of the page
\textheight=10.0in %How tall the text body is allowed to be on each page
\pagestyle{empty}
\begin{document}
\centerline{{ \LARGE \bf Problem Set 8}} 
\centerline{CSCI 3104 Spring 2014} 
\centerline{Cristobal Salazar}
\centerline{07/22}
\centerline{Partner: Alex Tsankov}

\noindent{\Large \bf Problem 1}
\\

\indent{\large a)
\\

\indent{\large b)
\\

\noindent{\Large \bf Problem 2}

\indent{\large a) }
\\

\indent{\large b) }
\\

\indent{\large c) }
\\


\noindent{\Large \bf Problem 3}
\\
\indent{\large a) If we have a graph G with four vertices:\{A, B, C, D\}, with weighted edges:\{AB:1, AC:12, BC: 1, CD: 2\}, we will have a unique minimum spanning tree. This graph has the followig two spanning trees from vertex A with edges: $T_1=$\{AB:1, BC:1, CD:2\} and $T_2=$\{AB:1,AC:12,CD:2\}. Even though there are two spanning trees, there is only one, unique MST.}
\\

\indent{\large b) For this problem, we must assume that edge weights do not have to be distinct. If all edge weights are distinct, then Golum would be correct. So, assuming we can have multiple edges with the same weight, the following undirected graph $G$ has two MSTs: \\vertices:\{A, B, C, D, E, F\}\\edges:\{AE:2, AB:4, BC:4, BD:4, DC:4, CF:2\} \\We can see that no edge is unique, and there is still two minimum spanning trees, therefore Golum's claim is incorrect.}
\\

\indent{\large c) For this claim, we must assume that edge weights do not have to be distinct. To prove the claim by contradiction we say that graph $G$ has two minimum spanning trees, $T_{MST}$ and $T'_{MST}$. Let edge $(u,v)$ be in the minimum spanning tree $T_{MST}$, but not $T'_{MST}$. If we were to remove the edge from $T_{MST}$, we will cut the tree into two partitions, $T_u$ and $T_v$. Next, let the edge $(a,b)$ be the unique minimum edge crossing the partition. Given $(x,y) \neq (u,v)$ and we know the weight of $(a,b)$$<$ the weight of $(u,v)$, therefore the spanning tree $(T_{MST}-(u,v))\cup(a,b)$ has a less weight than $T_{MST}$, which is a contradiction. Now, since the edge $(u,v)$ is not in $T'_{MST}$, we say that the path between $u$ and $v$ is path $l$.  Because path $l$ exists, we know that there is some edge, $e$, between $T'_u$ and $T'_v$ (A partition of $T'_{MST}$ where $u$ is in $T'_u$ and $v$ is  in$T'_v$). Now, we also know that $W(u,v)<W(e)$ because as stated above, $(u,v)$ is an unique minimum weighted edge. If we add $(u,v)$ to $T'_{MST}$ we get a cycle composed of the edge $(u,v)$ and the path $l$. By removing any edge from the cycle we get the
spanning tree $(T′_{MST} \cup (u,v)) − e$ which has a lower weight than $T′_{MST}$, this is also a contradiction. Therefore we know that given our claim, there can only be one, unique minimum spanning tree.}
\\

\indent{\large d) To do this, we will modify Kruskal's algorithm which takes $O(E*lg(V))$. We can take Kruskal's, and add some checks in the algorithm like so: 
\begin{verbatim}
uniqueMST(G){ 
  int x[] = sortEdges(G) //array of sorted edges
  tree MST = initialize tree
  boolean uniqueCycle = false
  boolean uniqueMinPart = false
  for(i=1; i < x.length; i++){
    if(createsCycle(MST, x[i])){
      if(isMaxUniqueEdge(x[i]))
        uniqueCycle = true
    }
    else{
      addToTree(x[i])
      if(isMinEdgeConnectingPartitions(MST, x[i]))
        uniqueMinPart = true
    }
  }
  if(uniqueCycle && uniqueMinPart)
    return true
  else
    return false
}
\end{verbatim}
This algorithm performs Kruskal's algorithm, while checking to see if our claim from above is satisfied, thus returning true if there is only one, unique MST.
}

\noindent{\Large \bf Problem 4}
\\
\indent{\large   }
\\

\noindent{\Large \bf Problem 5}
\\
\indent{\large a) }
\\

\indent{\large b) }
\\

\noindent{\Large \bf Sources}
\\
\indent $\bullet$ hhttp://www.math.uiuc.edu/~west/openp/pancake.html
%this is for 4c
\\
\indent $\bullet$ http://mypages.valdosta.edu/dgibson/courses/cs3410/notes/ch08.pdf
%this is for 4b


\end{document}